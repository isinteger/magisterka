% this file is called up by thesis.tex
% content in this file will be fed into the main document

\chapter{Podsumowanie} % top level followed by section, subsection	


% ----------------------- paths to graphics ------------------------

% change according to folder and file names
\ifpdf
    \graphicspath{{7/figures/PNG/}{7/figures/PDF/}{7/figures/}}
\else
    \graphicspath{{7/figures/EPS/}{7/figures/}}
\fi

Obecnie istnieje bardzo dużo rożnych usług oferujących przetwarzanie mowy. Jednak ich poważną wadą jest to, że dziedzina ich pracy jest bardzo wąska. Brakuje systemów oferujących więcej niż jeden sposób transformacji. Autorzy tej pracy starali się sprawdzić czy integracja takich systemów jest dobrym pomysłem oraz jeżeli tak to jakie podejście jest najlepsze. Jak zostało to zaprezentowane w powyższych rozdziałach stworzenie systemu integrującego usługi posiadające różne funkcje ma dużo zalet i bardzo ułatwia tworzenie aplikacji wykorzystujących te usługi. Co więcej dodatkowa warstwa abstrakcji, która powstaje w ten sposób, nie wpływa w sposób znaczący na szybkość działania aplikacji klienckich (w porównaniu z sytuacją w której aplikacja kliencka sama musiałaby komunikować się z tymi usługami). Wszystko to wyraźnie pokazuje, że istnienie systemu integrującego różne usługi przetwarzania mowy ma sens i jest na niego zapotrzebowanie. \\
Co do wyboru sposobu integracji, jak zostało to zaprezentowane wyżej, istnieje wiele odrębnych sposobów, z których każdy ma swoje zalety i wady. Po dogłębnej analizie dostępnych rozwiązań integracyjnych autorzy pracy zdecydowali, że najlepszy efekt daje najnowsze rozwiązanie a więc ESB. Posiada ono szereg zalet które należy uwypuklić:
\begin{itemize}
	\item Łatwość rozbudowy - gdy powstanie jakaś nowa usługa, oferująca ciekawą funkcjonalność jedynym wysiłkiem jaki należy wykonać w celu użycia go, jest wpięcie usługi do szyny
	\item Lekkość - warstwa abstrakcji nakładana przez ESB w celu połączenia dostępnych usług jest bardzo mała i nie ma wpływu na wydajność
	\item Skalowalność - różne instancje ESB mogą łączyć się razem w celu uzyskania wyższej wydajności oraz rozproszenia geograficznego
	\item Bogactwo interfejsów - wszystkie istniejące implementacje ESB oferują wiele różnych punktów końcowych, służących jako interfejsy wejścia/wyjścia, obsługujących wiele formatów
\end{itemize}
Wykorzystanie takiego podejścia daje też inny, duży plus, jakim niewątpliwie jest łatwość ukrywania implementacji. Klienci systemu opartego na takim rozwiązaniu nie będą wiedzieć z jakich konkretnie usług korzystają, przez co w razie wystąpienia kłopotów czy też pojawienia się nowszego, lepszego lub ciekawszego rozwiązania można taką usługę łatwo podmienić, bez konieczności zmieniania API. Kolejną dużą zaletą wykorzystania ESB, jest możliwość wykorzystania routingu. Dzięki temu sterowanie całym systemem i kolejnością wywoływania poszczególnych usług jest możliwa za pomocą prostego pliku xml.  \\
Zaprezentowana praca magisterska zagłębia się w dziedzinę, która jest dość nowatorska. Zaproponowane rozwiązanie i przykładowa implementacja pokazują zarówno, iż temat jest ciekawy i warty dalszych badań oraz, że istniejące w tej chwili rozwiązania integracyjne są odpowiednie dla specyficznej dziedziny jaką jest przetwarzanie mowy i można je z powodzeniem stosować. W rozdziale pierwszym przedstawiono listę kilku pytań, na które, na podstawie, tej pracy można udzielić odpowiedzi, tak więc:
\begin{itemize}
	\item czy modele i wzorce proponowane przy innych rozwiązaniach integracyjnych mają zastosowanie dla tak specyficznej dziedziny problemu? - tak, rozwiązania oparte o technologię ESB sprawdzają się bardzo dobrze
	\item czy istnieje zapotrzebowanie na platformę integrującą systemy przetwarzania mowy? - tak, istnieje wiele dziedzin w których można wykorzystać aplikacje bazujące na takiej platformie, co więcej przykładowe aplikacje pokazują, że jest ona funkcjonalna
	\item jakie mogą być zastosowania takiego zintegrowanego systemu? - wiele różnych zastosowań, opisane wyżej Automatyczne Dyktando, Lektor SMS, czy platforma Google Goggles 
	\item jakie są zalety istnienia takiej platformy? - uproszczenie interfejsu dla programisty aplikacji klienckich, łatwa możliwość konfiguracji, wiele zastosowań, innowacyjność
	\item jakie są wady takiej platformy? - wadami tej platformy są: dodatkowy koszt czasowy, narzuceni dostarczyciele usług, konieczność dzielenia danych wejściowych na małe paczki z powodu braku strumieniowania
\end{itemize}

% ----------------------- contents from here ------------------------






% ---------------------------------------------------------------------------
% ----------------------- end of thesis sub-document ------------------------
% ---------------------------------------------------------------------------