
% this file is called up by thesis.tex
% content in this file will be fed into the main document

%: ----------------------- introduction file header -----------------------
\chapter{Wstęp}

% the code below specifies where the figures are stored
\ifpdf
    \graphicspath{{1_introduction/figures/PNG/}{1_introduction/figures/PDF/}{1_introduction/figures/}}
\else
    \graphicspath{{1_introduction/figures/EPS/}{1_introduction/figures/}}
\fi

% ----------------------------------------------------------------------
%: ----------------------- introduction content ----------------------- 
% ----------------------------------------------------------------------



%: ----------------------- HELP: latex document organisation
% the commands below help you to subdivide and organise your thesis
%    \chapter{}       = level 1, top level
%    \section{}       = level 2
%    \subsection{}    = level 3
%    \subsubsection{} = level 4
% note that everything after the percentage sign is hidden from output



\section{Definicja problemu } % section headings are printed smaller than chapter names
% intro
Komputerowe przetwarzanie mowy jest w ostatnim czasie szybko rozwijającą się dziedziną w przemyśle informatycznym. Wykorzystuje się je w wielu różnych dziedzinach zarówno związanych z rozrywką ( rozmainte gry polegające na jak najwierniejszym zaśpiewaniu podanego utworu), z życiem codziennym (Apple Siri czyli wirtualny asystent) jak i z pracą(różne systemy telefoniczne działające bez udziału człowieka). W wielu sytuacjach wykorzystanie komputerowo generowanej mowy jest niezastąpione.\\
Podobnie sytuacja wygląda z wykorzystaniem architektury SOA. Jest to jedno z najważniejszych osiągnieć informatycznych ostatnich lat. Podejście to zdecydowanie dominuje w obecnie projektowanych architekturach. Jednym z głowych powodów takiego stanu rzeczy jest uproszczenie modelu współczesnych dużych aplikacji. Umożliwia ono podzielenie systemu na serwisy które są niezależnymi, samodzielnymi, łatwymi w zarządzaniu, posiadającymi jasno i jednoznacznie zdefiniowane interfejsy encjami, które można wykorzystać w różnej kolejności w celu otrzymania różnych efektów. \\
Celem tej pracy jest zbadanie możliwości wykorzystania nowoczesnej i prężnie rozwijającej się architektury do stworzenia gotowego systemy ułatwiającego i ujednolicającego proces tworzenia aplikacji przetwarzających mowę.
 


\subsection{Obszar badań} % subsection headings are again smaller than section names
W skład zakresu pracy wchodzą zarówno technologie związane z architekturą SOA jak i z przetwarzaniem mowy. W celu użycia odpowiednich narzędzi przeprowadzono wiele badań i testów w czasie których dużą wagę przykładano do:
\begin{itemize}
 	\item możliwości konfiguracyjnych
	\item wydajności
	\item łatwości rozbudowy
	\item dostępnej dokumentacjji i pomocy technicznej
\end{itemize}
W efekcie przeprowadzonych czynności okazało się, że najlepsze efekty uzyskać można korzystając z szyny ESB  Apache ServiceMix,  syntezatorów mowy Ivona i FreeTTS oraz innych mniej istotnych narzędzi potrzebnych do stworzenia architektury zgodnej z SOA oraz zdolnej do efektywnego wykonywania postawionych przed nią zadań. 


